\documentclass[10pt,a4paper]{article}
\usepackage[a4paper]{geometry}

\usepackage{polski}
\usepackage{xltxtra}
\usepackage{indentfirst}
\usepackage{relsize}
\usepackage{fancyvrb}
\usepackage{hyperref}
\hypersetup{
    pdftitle={Sprawozdanie z ćwiczenia nr 4 z laboratorium Programowanie internetowe},%
    pdfauthor={Tomasz Cudziło},%
    colorlinks=true,        % false: boxed links; true: colored links
    linkcolor=black,        % color of internal links
    citecolor=green,        % color of links to bibliography
    filecolor=magenta,      % color of file links
    urlcolor=cyan,          % color of external links
    unicode=true,           % non-Latin characters in Acrobat’s bookmarks
    pdfstartview={FitH},    % fits the width of the page to the window
    pdfnewwindow=true       % links in new window
}
\usepackage{paralist}

%% tweak fonts
\defaultfontfeatures{Mapping=tex-text}
\setromanfont{Charis SIL}
\setsansfont[Scale=MatchLowercase]{Helvetica Neue}
\setmonofont[Scale=MatchLowercase]{Menlo}
\linespread{1.25}

%% define custom commands and environments
\DefineVerbatimEnvironment%
  {SmallVerbatim}%
  {Verbatim}{fontsize=\relsize{-0.5},numbers=left,numbersep=-10pt,frame=lines,tabsize=4}

\newcommand{\f}[1]{\texttt{#1}}
\newcommand{\s}[1]{\textsf{#1}}

\newcommand{\rev}{35c18ee3786b53133ca699f3f44c15e18e2d7a25}
\newcommand{\revhref}[1] {\href{https://github.com/student-tomasz/pi-cwiczenie-4/blob/\rev/#1}{\f{#1}}}

\begin{document}

%%fakesection{Tytuł}
\title{
  Sprawozdanie z~ćwiczenia nr~4\\z~laboratorium Programowanie internetowe
}
\author{
  Tomasz Cudziło\\
  \textsc{PW EE Informatyka}\\[10pt]
}
\date{\today}
\maketitle



\section{Opis projektu}


Celem ćwiczenia było stworzenie aplikacji pozwalającej na upload/download plików
na/z serwera wykorzystując \f{PHP} i przechowując informacje o plikach w
\f{MySQL}.


\subsection{Kod źródłowy}

\begin{description}
  \item[Adres projektu:] \hfill \\
  \url{http://volt.iem.pw.edu.pl/~cudzilot/pi/cw4/}
  \item[Repozytorium projektu:] \hfill \\
  \url{https://github.com/student-tomasz/pi-cwiczenie-4}
  \item[Repozytorium sprawozdania:] \hfill \\
  \url{https://github.com/student-tomasz/pi-sprawozdanie-4}
\end{description}
Stan projektu na czas pisania sprawozdania:
\href{https://github.com/student-tomasz/pi-cwiczenie-4/tree/\rev}{\f{\rev}}.


\subsection{Biblioteki}

Wykorzystywane jest \emph{jQuery} \cite{jquery} do tworzenia odnośników do
walidatorów. Style \f{CSS} resetujące i normalizujące wszystkie elementy
pochodzą z \emph{YUI 3} \cite{yui3}. Pozostałe reguły \f{CSS} dotyczące
przycisków, formularzy i tabeli z plikami są silnie wzorowane na stylach z
\emph{Twitter Bootstrap} \cite{bootstrap}.


\subsection{Zgodność ze standardami}

\f{HTML} powinien przechodzić walidację dla zadeklarowanego standardu \f{XHTML
4.01 Strict}. \f{CSS} jest zgodny ze standardem \f{CSS 2.1}. Reguły opisujące
przyciski formularzy, wykorzystują animacje oraz gradienty. Należą one do
standardu \f{CSS 3}.

Napisany \f{JavaScript} wykorzystuje klasy i metody wchodzące dopiero do użytku,
wraz z ogólnie pojętym \f{HTML 5}. Nie miałem okazji poznać nowych możliwości
\f{HTML 5}, więc wykorzystałem część z nich w tym projekcie. W dalszych
paragrafach szczegółowo opisuję wykorzystane klasy.

Ponieważ rozważane specyfikacje są młode strona działa poprawnie tylko w Mozilla
Firefox 8.0+, Google Chrome 15+ oraz Safari 5.1+.



\section{Wykonanie}


\subsection{Funkcje pomocnicze}

Pliki \revhref{config.php}, \revhref{connect.php} i \revhref{disconnect.php}
zawierają instrukcje pomocnicze.

\revhref{config.php} tworzy \f{Array} ze zmiennymi potrzebnymi do połączenia z
bazą danych i ścieżką docelową zapisu przesyłanych plików. Pliki
\revhref{connect.php} i \revhref{disconnect.php} zawierają odpowiednio kod
otwierający i kończący połączenie z bazą danych.

Plik \revhref{sanitize.php} definiuje funkcję pomocniczą \f{sanitize()}, która
jest prymitywnym sposobem na ignorowanie znaków. Zapewnia ona ignorowanie znaków
mogących w niepożądany sposób wpłynąć na polecenia \f{PHP} lub zapytania
\f{SQL}.


\subsection{Lista plików}

Za wyświetlanie listy plików, znajdujących się na serwerze, odpowiadają pliki
\revhref{index.html}, \revhref{index.php} oraz \revhref{javascripts/updater.js}.

Plik \revhref{index.html} jest statyczny. Zawiera formularz do wysyłania
plików i pustą tabelę przeznaczoną na informacje o plikach na serwerze.

Plik \revhref{index.php} służy do pobierania informacji o plikach znajdujących
się na serwerze. Tworzone jest połączenie z bazą danych poprzez wczytanie pliku
\revhref{connect.php}, następnie wykonywane jest zapytanie pobierające wszystkie
wiersze z tabeli ćwiczenia. Iterując po otrzymanych wierszach, wypisywane są
tagi \f{<tr>...</tr>} przeznaczone do tabeli w \revhref{index.html}.

Oba te pliki łączy obiekt pseudo-klasy \f{Updater} zdefiniowany w pliku
\revhref{javascripts/updater.js}. Prototyp \f{Updater} posiada metodę
\f{update()}, która wysyła asynchronicznie żądanie \f{HTTP GET} do
\revhref{index.php}. Serwer wypisuje jako treść odpowiedzi, opisany wyżej kod
\f{HTML}. Gdy żądanie zostanie zakończone i status odpowiedzi jest równy \f{200
OK} to otrzymany tekst, jest wstawiany do środka tabeli.

Metoda \f{update()} jest wywoływana, gdy strona zostanie załadowana oraz gdy
wysyłanie pliku zakończy się sukcesem.


\subsection{Wysyłanie plików}

Wysyłanie plików odbywa się poprzez formularz w pliku \revhref{index.html},
obsługę zapytania w \revhref{upload.php} oraz wysłanie zapytania przez
\revhref{javascripts/uploader.js}.

\subsubsection{Przesyłanie plików przez \f{JavaScript}}

Klasa \f{Uploader} posiada dwie metody: \f{fileSubmitted} oraz \f{fileChanged}.
Metoda \f{fileChanged} jest wywoływana gdy zawartość pola wyboru pliku ulega
zmianie. Blokuje ona możliwość wysłania pliku gdy, wybrany plik przekracza limit
wielkości lub plik nie został wybrany. Metoda \f{fileSubmitted} jest wywoływana
gdy użytkownik zdecyduje się wysłać plik. Jej schemat działania przedstawia blok
pseudokodu:

\begin{SmallVerbatim}
    zignoruj zdarzenie
    przestań propagować zdarzenie

    if plik istnieje
        zablokuj przycisk formularza
        stworz obiekt zapytania asynchronicznego
        uzupelnij naglowki zapytania
        uzupelnij funkcje reagujace na zakonczenie zapytania
        wyslij zapytanie
    end
\end{SmallVerbatim}

Wyżej opisany kod korzysta z klas \f{JavaScript} opisanych w młodych
specyfikacjach. Obiekt zapytania jest obiektem klasy \f{XMLHttpRequest} opisanej
w specyfikacji \f{XMLHttpRequest Level 2} \cite{xhr}. Instancja pliku --- obiekt
klasy \f{File} pochodzi ze specyfikacji \f{FileAPI} \cite{fileapi}.

Ich połączenie pozwala na przesyłanie plików jako treść zapytania. Pozostałe
dane o pliku --- jego nazwa i typ są przesyłane oddzielnie, w nagłówku
zapytania. Wysyłanie pliku jest wykonywane asynchronicznie. Zapytanie dostaje
anonimowe funkcje uruchamiane w przypadku powodzenia lub wystąpienia błędu.

\subsubsection{Odbiór pliku przez serwer}

Plik \revhref{upload.php} zachowuje się analogicznie do \revhref{index.php}.
Inicjalizowane jest połączenie z bazą danych, dane z nagłówka są wstawiane do
zapytania \f{SQL}. Przy pomyślnym wykonaniu utworzonego zapytania \f{SQL},
tworzony jest nowy wiersz w tablicy, a plik zostaje zapisany na dysku.

Odczytanie zawartości przesłanego pliku odbywa się w funkcji
\f{file\_get\_contents('php://input')}. Argumentem jest nazwa specjalnego
strumienia \f{php://input}, który w tym przypadku wskazuje na ciało
\emph{(body)} zapytania.


\subsection{Konwersja plików}

Konwersja plików odbywa się za pomocą trzech plików: \revhref{convert.html},
\revhref{convert.php} i \revhref{javascripts/converter.js}.

Akcja konwersji kodowania pliku jest dostępna tylko dla plików tekstowych.
Rozpoznanie typu pliku następuje bezpośrednio przed przesłaniem plików na
serwer, za pomocą atrybutu \f{type} w obiektach klasy \f{File} z nowego
\f{FileAPI} \cite{fileapi}. Typ pliku zapamiętywany jest w bazie danych.

Po wyborze akcji \emph{konwertuj} pojawia się okno dialogowe, gdzie użytkownik
wybiera kodowanie źródłowe i końcowe dla pliku, oraz potwierdza wykonanie lub
anulowanie operacji. Oknem dialogowym zarządza metoda \f{Converter.convert(id)}
zdefiniowana w \revhref{javascripts/converter.js}. Do pokazania okna dialogowego
wykorzystywana jest, wchodząca do standardu, metoda \f{window.showModalDialog()}
\cite{modal}.

Do okna dialogowego przekazywane są id i nazwa pliku, a zwracane są żądane
kodowania lub \f{false}, jeśli użytkownik anuluje/zamknie okno dialogowe.

Posiadając id pliku, oraz kodowanie źródłowe i końcowe wysyłane jest zapytanie
\f{HTTP POST} do \revhref{convert.php}, gdzie wywoływana jest funkcja \f{iconv}
z podanymi argumentami.


\subsection{Usuwanie plików}

Usuwanie plików działa identycznie jak przesyłanie plików na serwer. Różnicą
jest tylko treść i cel wysyłanego zapytania. Obiekt zapytania tworzony jest w
\revhref{javascripts/destroyer.js}. Metoda \f{destroy()} wysyła zapytanie typu
\f{POST} do \revhref{destroy.php}. Zawartością zapytania jest instancja klasy
\f{FormData} \cite{formdata} zawierająca id pliku.

Po otrzymaniu zapytania, \revhref{destroy.php} sprawdza, czy plik istnieje, a
następnie usuwa wiersz z tablicy oraz plik z dysku.


\subsection{Pobieranie plików}

Pobieranie plików polega na wysłaniu zapytania \f{GET}. Standardowo, przez
kliknięcie odnośnika w oknie przeglądarki. Zwracana jest odpowiedź z ręcznie
spreparowanym nagłówkiem, informującym przeglądarkę, że przesyłany jest plik.
Ciałem odpowiedzi jest żądany plik. Jeżeli plik o danym \f{id} nie istnieje,
zwracana jest pusta odpowiedź z kodem \f{404 Not Found}.



\section{Wnioski}

Tworzenie nawet prostych aplikacji w \f{PHP} bez wsparcia bibliotek zewnętrznych
albo frameworków takich jak \emph{CakePHP} jest dosyć uciążliwe.

Konieczne jest ciągłe walidowanie danych wprowadzane przez użytkownika.
Zarządzanie bazą danych jest uciążliwe, trzeba nawiązywać połączenie za każdym
razem, a przy zmianie silnika bazy danych upewnić się, że zapytania \f{SQL} będą
działały poprawnie.

Zaimponowały mi nowe możliwości udostępniane przez \f{JavaScript}, zwłaszcza
obsługa plików po stronie klienta, oraz niemal natywne okna dialogowe o dowolnej
zawartości.



\begin{thebibliography}{9}
  \bibitem{jquery}
    \url{http://jquery.com/}
  \bibitem{yui3}
    \url{http://yuilibrary.com/yui/docs/cssreset/}
  \bibitem{bootstrap}
    \url{http://twitter.github.com/bootstrap/}
  \bibitem{xhr}
    \url{http://www.w3.org/TR/XMLHttpRequest2/#the-send-method}
  \bibitem{fileapi}
    \url{http://www.w3.org/TR/FileAPI/}
  \bibitem{modal}
    \url{https://developer.mozilla.org/en/DOM/window.showModalDialog}
  \bibitem{formdata}
    \url{http://www.w3.org/TR/XMLHttpRequest2/#the-formdata-interface}
\end{thebibliography}



\end{document}
