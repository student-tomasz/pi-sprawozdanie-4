\documentclass[10pt,a4paper]{article}
\usepackage[a4paper]{geometry}

\usepackage{polski}
\usepackage{xltxtra}
\usepackage{indentfirst}
\usepackage{relsize}
\usepackage{fancyvrb}
\usepackage{hyperref}
\hypersetup{
    pdftitle={Sprawozdanie z ćwiczenia nr 4 z laboratorium Programowanie internetowe},%
    pdfauthor={Tomasz Cudziło},%
    colorlinks=true,        % false: boxed links; true: colored links
    linkcolor=black,        % color of internal links
    citecolor=green,        % color of links to bibliography
    filecolor=magenta,      % color of file links
    urlcolor=cyan,          % color of external links
    unicode=true,           % non-Latin characters in Acrobat’s bookmarks
    pdfstartview={FitH},    % fits the width of the page to the window
    pdfnewwindow=true       % links in new window
}
\usepackage{paralist}

%% tweak fonts
\defaultfontfeatures{Mapping=tex-text}
\setromanfont{Charis SIL}
\setsansfont[Scale=MatchLowercase]{Helvetica Neue}
\setmonofont[Scale=MatchLowercase]{Menlo}
\linespread{1.25}

%% define custom commands and environments
\DefineVerbatimEnvironment%
  {SmallVerbatim}%
  {Verbatim}{fontsize=\relsize{-0.5},numbers=left,numbersep=-10pt,frame=lines,tabsize=4}

\newcommand{\f}[1]{\texttt{#1}}
\newcommand{\s}[1]{\textsf{#1}}

\newcommand{\rev}{35c18ee3786b53133ca699f3f44c15e18e2d7a25}
\newcommand{\revhref}[1] {\href{https://github.com/student-tomasz/pi-cwiczenie-4/blob/\rev/#1}{\f{#1}}}

\begin{document}

%%fakesection{Tytuł}
\title{
  Sprawozdanie z~ćwiczenia nr~4\\z~laboratorium Programowanie internetowe
}
\author{
  Tomasz Cudziło\\
  \textsc{PW EE Informatyka}\\[10pt]
}
\date{\today}
\maketitle



\section{Opis projektu}

Celem ćwiczenia było stworzenie aplikacji pozwalającej na upload/download plików
na/z serwera wykorzystując \f{PHP} i przechowując informacje o plikach w
\f{MySQL}.

\subsection{Kod źródłowy}
\begin{description}
  \item[Adres projektu:] \hfill \\
  \url{http://volt.iem.pw.edu.pl/~cudzilot/pi/cw4/}
  \item[Repozytorium projektu:] \hfill \\
  \url{https://github.com/student-tomasz/pi-cwiczenie-4}
  \item[Repozytorium sprawozdania:] \hfill \\
  \url{https://github.com/student-tomasz/pi-sprawozdanie-4}
\end{description}
Stan projektu na czas pisania sprawozdania:
\href{https://github.com/student-tomasz/pi-cwiczenie-4/tree/\rev}{\f{\rev}}.

\subsection{Biblioteki}
Wykorzystywane jest \emph{jQuery} \cite{jquery} do stworzenia odnośników do
walidatorów kodu strony. Style \f{CSS} resetujące i normalizujące wszystkie
elementy pochodzą z \emph{YUI 3} \cite{yui3}. Pozostałe reguły \f{CSS} dotyczące
przycisków, formularzy i tabeli z plikami są silnie wzorowane na stylach z
\emph{Twitter Bootstrap} \cite{bootstrap}.

\subsection{Zgodność ze standardami}
\f{HTML} powinien przechodzić walidację dla zadeklarowanego standardu \f{XHTML
4.01 Strict}. \f{CSS} jest zgodny ze standardem \f{CSS 2.1}. Reguły opisujące
przyciski formularzy, wykorzystują animacje oraz gradienty. Należą one do
standardu \f{CSS 3}.

Napisany \f{JavaScript} wykorzystuje klasy i metody wchodzące dopiero do użytku,
wraz z ogólnie pojętym \f{HTML 5}. Nie miałem okazji poznać nowych możliwości
\f{HTML 5}, więc wykorzystałem część z nich w tym projekcie. W dalszych
paragrafach szczegółowo opisuję wykorzystane klasy.

Ponieważ rozważane specyfikacje są młode strona działa poprawnie tylko w Mozilla
Firefox 8.0+, Google Chrome 15+ oraz Safari 5.1+.



\begin{thebibliography}{9}
  \bibitem{jquery}
    \url{http://jquery.com/}
  \bibitem{yui3}
    \url{http://yuilibrary.com/yui/docs/cssreset/}
  \bibitem{bootstrap}
    \url{http://twitter.github.com/bootstrap/}
\end{thebibliography}



\end{document}
